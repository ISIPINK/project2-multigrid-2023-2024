\documentclass[a4paper,12pt]{article}

\setlength{\textwidth}{15.0cm}
\setlength{\textheight}{24.0cm}
\setlength{\topmargin}{0cm}
\setlength{\headsep}{0cm}
\setlength{\headheight}{0cm}
\pagestyle{plain}

\usepackage{hyperref}
\hypersetup{
    colorlinks=true,
    linkcolor=blue,
    filecolor=magenta,      
    urlcolor=blue,
    citecolor=blue,
    linktoc=page
}
\usepackage[dvips]{epsfig}
\usepackage{tikz}
\usepackage[english]{babel}
\usepackage{caption}
\captionsetup{font=it}
\usepackage[autostyle, english=american]{csquotes}

\usepackage[
backend=biber,
style=alphabetic,
]{biblatex}

\renewcommand{\bibfont}{\footnotesize}

\usepackage{amsmath,amssymb,amsthm}

\usepackage{comment}
\usepackage{listings}

% \addbibresource{bibliography2.bib} 
\setlength{\parindent}{0pt}

\selectlanguage{english}
\begin{document}

\title{Project 2023-2024: Simulation of the Double Slit Experiment
    and simulation of Wi-Fi signals at home}
\author{Isidoor Pinillo Esquivel}
\date{\today}
\maketitle


\section{Exterior Complex Scaling (ECS) boundary conditions}
\subsection{Equivalence of complex grid and complex wave number}

For homogenous the discretized Helmholtz equation ECS is equivalent to
complex wave number. Let $h$ be normal "real" grid spacing and $\tilde{h} = z h, z \in \mathbb{C} $
be the complex grid spacing.
Let $\sigma$ be normal real wave number and $\tilde{\sigma} = z^{2} \sigma$ be the complex wave number.
Let $u$ be the solution to the discretized Helmholtz equation with complex wave number on a normal grid
and $\tilde{u}$ be the solution to the discretized Helmholtz equation on the complex grid.

\begin{align}
    \frac{\tilde{u}(\tilde{x}-\tilde{h}) -2 \tilde{u}(\tilde{x}) + \tilde{u}(\tilde{x} - \tilde{h})}{\tilde{h}^2} + \sigma \tilde{u} & = 0  \Leftrightarrow \\
    \frac{\tilde{u}(\tilde{x}-zh) -2 \tilde{u}(\tilde{x}) + \tilde{u}(\tilde{x} - zh)}{z^{2}h^2} + \sigma \tilde{u}                  & = 0  \Leftrightarrow \\
    \frac{u(x-h) -2 u(x) + u(x - h)}{h^2} + z^{2} \sigma u                                                                           & = 0  \Leftrightarrow \\
    \frac{u(x-h) -2 u(x) + u(x - h)}{h^2} + \tilde{\sigma} u                                                                         & = 0
\end{align}

It choosing $z \sim e^{i \frac{\pi}{6} }$ is equivalent to choosing $\beta = 0.5$ because of the squaring.
This equivalence doesn't hold when there is a source term.
\subsection{ non uniform helmhotz matrix}

$$
    (\Delta_{h}u)_{i} = \left( \frac{u_{i+1} - u_{i}}{h_{i+1/2}} - \frac{u_{i} - u_{i-1}}{h_{i-1/2}} \right) \frac{2}{h_{i+1/2} +h_{i-1/2}}
$$

\end{document}
